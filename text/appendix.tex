\documentclass[twoside]{article}
\setlength{\oddsidemargin}{0.25 in}
\setlength{\evensidemargin}{-0.25 in}
\setlength{\topmargin}{-0.6 in}
\setlength{\textwidth}{6.5 in}
\setlength{\textheight}{8.5 in}
\setlength{\headsep}{0.75 in}
\setlength{\parindent}{0 in}
\setlength{\parskip}{0.1 in}
\usepackage{subfig}
\usepackage{amsmath,amsfonts,graphicx}

\newcounter{lecnum}
\renewcommand{\thepage}{\thelecnum-\arabic{page}}
\renewcommand{\thesection}{\thelecnum.\arabic{section}}
\renewcommand{\theequation}{\thelecnum.\arabic{equation}}
\renewcommand{\thefigure}{\thelecnum.\arabic{figure}}
\renewcommand{\thetable}{\thelecnum.\arabic{table}}
\newcommand{\mysection}[1]{\addtocounter{section}{1}\section*{#1}}

\begin{document}
\mysection{Decomposition by tumor size, cancer-specific mortality rates, and competing causes of death}

Let $\pi_i(t)$ and $e_i(x,t)$ be the proportion of patients and the life expectancy for cancer patients with tumor size $i$ (i.e., $<1$ cm, 1-2 cm, 2-3 cm, 3-5 cm, and 5+ cm.) at age $x$ and time $t$, 
respectively. That is, 
$e_i(x,t)$ represents tumor-size-specific life expectancy. The overall life expectancy at age $x$ time $t$ is given by
\begin{eqnarray}
  e(x,t)=\sum_{i=1}^{5}\,\pi_i(t)\,e_i(x,t) \notag
\end{eqnarray}
where $\sum_{i=1}^{5}\,\pi_i=1$.  Consider tumor size $i$.  The change
in life expectancy at age $x$ between times $t_1$ and $t_2$ can be
decomposed as (Kitigawa 1955): 
\begin{equation}
 e_i(x,t_2)-e_i(x,t_1) = \sum_{j=1}^k \int_x^\infty \left[ p_{i,j}(s,t_2)- p_{i,j}(s,t_1)\right] \left[ \frac{p_{i,-j}(s,t_1)+p_{i,-j}(s,t_2)}{2} \right]\,ds, \notag
\end{equation}
where $p_{i,j}(s,t)$ is the probability of surviving from birth to age
$s$ for cancer patients with tumor size $i$ cause $j$ at time $t$, and
$p_{i,-j}(s,t)$ is the analogue survival probability for all other
causes of death (other than $j$).  Similarly, the change in life
expectancy across age tumor sizes can be decomposed as: 
\begin{eqnarray}
  e(x,t_2)-e(x,t_1)&=&\sum_{i=0}^{3}\left[\,\pi_i(t_2)\,e_i(x,t_2)- \pi_i(t_1)\,e_i(x,t_1)\right] \notag \\
  &=&\sum_{i=0}^{3}\left[\pi_i(t_2)-\pi_i(t_1) \right]\left[\frac{e_i(x,t_1)+e_i(x,t_2)}{2}\right]+\sum_{i=0}^{3}\left[e_i(x,t_2)-e_i(x,t_1) \right]\left[\frac{\pi_i(t_1)+\pi_i(t_2)}{2}\right]. \notag
\end{eqnarray}

The above equation quantifies how much of the change in life
expectancy at age $x$ between times $t_1$ and $t_2$ is due to: [a]
shifts in the share of cancer tumor size (first term) and [b] changes
in tumor-size-specific life expectancy (second term).

%Formula \eqref{eqn:basic} would produce 6 terms (one for each cancer-stage proportion and for each cancer-stage mortality), which are the pieces you plotted as blue and red bars. 

We can further decompose the second term in the above equation by
cause of death. In doing so, we can quantify how much of this change
in cancer-specific life expectancy, $e_i(x,t_2)-e_i(x,t_1)$, is due to
improvements in cancer mortality and non-cancer mortality.  Using the
approach developed in Beltr\'{a}n-S\'{a}nchez et al. (2008),
\begin{eqnarray}
e_i(x,t_2)-e_i(x,t_1)=\sum_{j=1}^{2} \sum_{s=x}^{\omega}\left[L_{s,i,j}(t_2)-L_{s,i,j}(t_1) \right] \left[\frac{L_{s,i,-j}(t_2)+L_{s,i,-j}(t_1) }{2n} \right]
\label{eqn:causedecomp}
\end{eqnarray}
where $i$ corresponds to tumor size, $j$ is cause-specific mortality
among patients diagnosed with tumor size $i$, $s$ is age, $\omega$ is
the starting age of the oldest age interval, $n$ is the width of the age
interval, and $L_s$ are person-years lived in the life table.

%For simplicity, I think we only want to look at two causes of death among cancer patients: deaths due to cancer $i$ and all-other causes. That is, we are interested in finding the contribution of improvements in cancer mortality $i$ among those diagnosed with this type of cancer.

%Thus, we don't need additional proportions in equation \eqref{eqn:basic} for the additional decomposition, we just need to estimate equation \eqref{eqn:causedecomp} and use these results in the second term of equation \eqref{eqn:basic}.

We perform the decomposition starting at age 40, so the final decomposition equation is given by:
\begin{eqnarray}
  e(40,t_2)-e(40,t_1)&=&\sum_{i=1}^{5}\left[\,\pi_i(t_2)\,e_i(40,t_2)- \pi_i(t_1)\,e_i(40,t_1)\right] \notag \\
  &=&\sum_{i=1}^{5}\left[\pi_i(t_2)-\pi_i(t_1) \right]\left[\frac{e_i(40,t_1)+e_i(40,t_2)}{2}\right]+\sum_{i=1}^{5}\left[\mathtt{Diff} \right]\left[\frac{\pi_i(t_1)+\pi_i(t_2)}{2}\right] 
  %\label{eqn:basic}
\end{eqnarray}
where $\mathtt{Diff}$ is given by \eqref{eqn:causedecomp} evaluated at x=40.

\mysection{Decomposition by tumor size, cancer-specific mortality rates, and competing causes of death for 40-49 years old}
Let $\pi_{i,x}(t)$ be the proportion of cancer patients with tumor size $i$ (i.e., $<1$ cm, 1-2 cm, 2-3 cm, 3-5 cm, and 5+ cm.). These proportions can be computed by age, (e.g., ages 40-49 and 50 or older) so that $\pi_i(t)=\pi_{i,40-49}(t)+\pi_{i,50+}(t)$.  
\begin{eqnarray}
\sum_{i=1}^{5}\,\pi_i(t)&=&\sum_{i=1}^{5}\pi_{i,40-49}(t)+\pi_{i,50+}(t)=1  
\end{eqnarray}

Then, the change in life expectancy at age $x$ between times $t_1$ and $t_2$ can be estimated as:
{\footnotesize
\begin{eqnarray}
  e(x,t_2)-e(x,t_1)&=&\sum_{i=1}^{5}\left[\,\pi_i(t_2)\,e_i(x,t_2)- \pi_i(t_1)\,e_i(x,t_1)\right] \notag\\
  &=& \sum_{i=1}^{5}\left\{[\pi_{i,40-49}(t_2)+\pi_{i,50+}(t_2)]\,e_i(x,t_2)- [\pi_{i,40-49}(t_1)+\pi_{i,50+}(t_1)]\,e_i(x,t_1)\right\}\notag \\
  &=&\sum_{i=1}^{5}\left\{\pi_{i,40-49}(t_2)\,e_i(x,t_2)- \pi_{i,40-49}(t_1)\,e_i(x,t_1)\right\} + \sum_{i=1}^{5}\left\{\pi_{i,50+}(t_2)\,e_i(x,t_2)- \pi_{i,50+}(t_1)\,e_i(x,t_1)\right\} \notag 
  \label{eqn:basic}
\end{eqnarray}
}
Each summation in the above equation can be written as (Kitawaga (1955)):
\begin{multline}
e(x,t_2)-e(x,t_1)= \\
\sum_{i=1}^{5}\left[\pi_{i,40-49}(t_2)-\pi_{i,40-49}(t_1) \right]\left[\frac{e_i(x,t_1)+e_i(x,t_2)}{2}\right]+\sum_{i=1}^{5}\left[e_i(x,t_2)-e_i(x,t_1) \right]\left[\frac{\pi_{i,40-49}(t_1)+\pi_{i,40-49}(t_2)}{2}\right] +\\
\sum_{i=1}^{5}\left[\pi_{i,50+}(t_2)-\pi_{i,50+}(t_1) \right]\left[\frac{e_i(x,t_1)+e_i(x,t_2)}{2}\right]+\sum_{i=1}^{5}\left[e_i(x,t_2)-e_i(x,t_1) \right]\left[\frac{\pi_{i,50+}(t_1)+\pi_{i,50+}(t_2)}{2}\right] \\
\phantom{e(x,t_2)-e(x,t_1)}= \sum_{i=1}^{5}\left[\pi_{i,40-49}(t_2)-\pi_{i,40-49}(t_1) \right]\left[\frac{e_i(x,t_1)+e_i(x,t_2)}{2}\right] + 
\sum_{i=1}^{5}\left[\pi_{i,50+}(t_2)-\pi_{i,50+}(t_1) \right]\left[\frac{e_i(x,t_1)+e_i(x,t_2)}{2}\right] + \\
\sum_{i=1}^{5}\left[e_i(x,t_2)-e_i(x,t_1) \right]\left[\frac{\pi_i(t_1)+\pi_i(t_2)}{2}\right]
\label{agedec}
\end{multline}
The first two terms of equation \eqref{agedec} correspond to the contribution of changes in the share of tumor size among people aged 40-49 and 50+ to changes in cancer life expectancy between times 1 and 2. We can additionally estimate the contribution of cancer-specific mortality rates to changes in life expectancy by age. The last term of \eqref{agedec} can be written as (see equation \eqref{eqn:causedecomp}):
\begin{multline}
e_i(40,t_2)-e_i(40,t_1)=\sum_{j=1}^{k} \sum_{s=40}^{49}\left[L_{s,i,j}(t_2)-L_{s,i,j}(t_1) \right] \left[\frac{L_{s,i,-j}(t_2)+L_{s,i,-j}(t_1) }{2n} \right] + \\
\sum_{j=1}^{k} \sum_{s=50}^{\omega}\left[L_{s,i,j}(t_2)-L_{s,i,j}(t_1) \right] \left[\frac{L_{s,i,-j}(t_2)+L_{s,i,-j}(t_1) }{2n} \right]
\end{multline}

\mysection{Assuming constant mortality within age intervals}
Let $M_{x,x+n}$ represent the mortality rate between ages $x$ and $x+n$. Then 
\begin{equation}
l_{x+n}=e^{-\int_x^{x+n}\mu(s)\,ds}=e^{-n\,M_{x,x+n}}
\end{equation}
We can then estimate the person-years lived between ages $x$ and $x+n$ as
\begin{equation}
_nL_{x}=l_x\,\int_x^{x+n} e^{-M_{x,x+n}(s-x)} ds=l_x \left(\frac{-1}{M_{x,x+n}}(e^{-n\,M_{x,x+n}}-1) \right)
\label{Lx}
\end{equation}
so, if we have 5-year age groups, then equation \eqref{Lx} would look like
\begin{equation}
_5L_{x}=l_x \left(\frac{-1}{M_{x,x+5}}(e^{-5\,M_{x,x+5}}-1) \right)
\end{equation}
For the last age group (open-ended, say 100+), we can assume there are no person-years lived beyond a certain time (say no more than 10yrs) to compute $_+L_{100}$ as
\begin{equation}
_+L_{100}=l_{100} \left(\frac{-1}{M_{100+}}(e^{-10\,M_{100+}}-1) \right)
\end{equation}

\end{document}





